 \documentclass[aspectratio=169]{beamer}

% Must be loaded first
\usepackage{tikz}
\usetikzlibrary{shapes,arrows,positioning,fit,calc}

\usepackage[utf8]{inputenc}
\usepackage{textpos}

% Font configuration
\usepackage{fontspec}

\usepackage[listings, minted]{tcolorbox}
\usepackage{smartdiagram}
\usepackage[export]{adjustbox}

\input{font.tex}

% Tikz for beautiful drawings
\usetikzlibrary{mindmap,backgrounds}
\usetikzlibrary{arrows.meta,arrows}
\usetikzlibrary{shapes.geometric}

% Minted configuration for source code highlighting
\usepackage{minted}
\setminted{highlightcolor=orange!50, linenos}
\setminted{style=lovelace}

\usepackage[listings, minted]{tcolorbox}
\tcbset{left=6mm}

% Use the include theme
\usetheme{codecentric}

% Metadata
\title{Textanalysis with Monoids}
\author{Markus Hauck (@markus1189)}

% The presentation content
\begin{document}

\begin{frame}[noframenumbering,plain]
  \titlepage{}
\end{frame}

\section{Introduction}\label{sec:introduction}

\begin{frame}
  \begin{center}
    {\Huge Textanalysis with Monoids\\}
    (A taste of FP)
  \end{center}
\end{frame}

\begin{frame}
  \frametitle{Introduction}
  \begin{itemize}
  \item IT Consultant at codecentric
  \item working in Scala since >5 years
  \item passionate functional programmer
  \end{itemize}
\end{frame}

\section{Prelude}\label{sec:prelude}

\begin{frame}
  TODO: Lego vs Duplo?
\end{frame}

% re-use some slides from /home/markus/repos/topics/old/scala-io-applying-fp/tex/applying-fp.tex

\begin{frame}
  \frametitle{Typeclasses}
  \begin{itemize}
  \item forget the ``class'' part again (too overloaded)
  \item a way to implement overloaded functions
  \item not (yet) first class in scala
  \item encoded as class/trait with abstract methods
  \item use implicit resolution to define and pass around instances
  \end{itemize}
\end{frame}

\begin{frame}[fragile]
  \frametitle{Typeclasses}
  \begin{minted}{scala}
    abstract class Showable[A] {
      def show(x: A): String
    }
  \end{minted}
  \begin{minted}{scala}
    object Showable {
      implicit val showableInt = new Showable[Int] {
        def show(x: Int): String = x.toString
      }
    }
  \end{minted}
\end{frame}

\begin{frame}
  \frametitle{Typeclasses}
  \begin{itemize}
  \item disclaimer: if you want to nitpick, ``Int is a TC'' is wrong
  \item {State,Option,List} is \textbf{not} a Monad
  \item correct: {State,Option,List} has a (valid) Monad instance
  \end{itemize}
\end{frame}

\begin{frame}
  \frametitle{Typeclasses \textemdash{} Laws}
  \begin{itemize}
  \item typeclasses need laws
  \item otherwise it is super hard to reason about code
  \item at least without knowing all the instances (impossible)
  \item that's why custom typeclasses without laws are frowned upon
  \item there are still reasons, but mostly: don't unless you know why
  \end{itemize}
\end{frame}

\section{Monoids}\label{sec:monoids}

\begin{frame}
  \frametitle{Monoids}
  \begin{itemize}
  \item claim: most of you are using them without knowing it
  \item \textbf{very} common pattern
  \item implemented as a typeclass + instances
  \item provided by FP libraries (cats)
  \end{itemize}
\end{frame}

\begin{frame}
  \frametitle{Monoids}
  \begin{itemize}
  \item spot the pattern:
  \end{itemize}
  \begin{minted}{scala}
       0  +       1  +       5
       1  *       2  *       5
      ""  + "Hello"  + "World"
  List() ++ List(4) ++ List(2)
  \end{minted}
\end{frame}

\begin{frame}
  \frametitle{Composition}
  \begin{itemize}
  \item probably the biggest reason for FP
  \item concepts actually compose
  \item let's see how monoids compose
  \end{itemize}
\end{frame}

\begin{frame}
  \frametitle{Composition of Monoids}
  \begin{itemize}
  \item nesting monoids
  \item \texttt{Option[A]} is a Monoid if \texttt{A} is a Monoid
  \end{itemize}
\end{frame}

\begin{frame}
  \frametitle{Product of Monoids}
  \begin{itemize}
  \item given a Monoid for \texttt{A} and a Monoid for \texttt{B}
  \item we can define a Monoid for \texttt{(A,B)}
  \item just use corresponding instances
  \end{itemize}
\end{frame}

\begin{frame}
  \frametitle{From Monoids to Folds}
  \begin{itemize}
  \item spot the pattern again:
  \end{itemize}
  \begin{minted}{scala}
    List(1,2,3).foldLeft(0)(_+_)
    List(1,2,3).foldLeft(1)(_*_)
    List("a", "b", "c").foldLeft("")(_+_)
  \end{minted}
\end{frame}

\begin{frame}
  \frametitle{From Monoids to Folds}
  \begin{itemize}
  \item our text analysis works with anything that has a fold
  \item using Spark? Well let's see
  \end{itemize}
\end{frame}

\begin{frame}[fragile]
  \frametitle{RDDs and Folds}
\begin{minted}{scala}
abstract class RDD[T] {
  /**
   * Aggregate the elements of each partition,
   * and then the results for all the partitions,
   * using a given associative function and a
   * neutral "zero value".
   */
  def fold(zeroValue: T)(op: (T, T) => T): T
}
\end{minted}
\end{frame}

\begin{frame}[fragile]
  \frametitle{Monoidal RDDs}
\begin{minted}{scala}
implicit class MonoidRDD[T](val rdd: RDD[T]) {

  // avoid conflicts with fold/reduce etc
  def combine(implicit M: Monoid[T]): T =
    rdd.fold(M.empty)(M.combine(_,_))

}
\end{minted}
\end{frame}

\begin{frame}[fragile,fragile]
  \frametitle{The Program}
\begin{minted}{scala}
val sc: SparkContext = ???
val file: String = ???

val data = sc.textFile(file).  // read the file
  flatMap(_.split("""\W+""")). // split into words
  map(expand)                  // action!

def expand(w: String) = (1, w.length, Map(w -> 1))

val (words,chars,wordMap) = data.combine
\end{minted}
\end{frame}

\begin{frame}
  \frametitle{Conclusion}
  \begin{itemize}
  \item TODO
  \end{itemize}
\end{frame}

\end{document}
